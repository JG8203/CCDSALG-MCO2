\section*{Results and Discussion}
\addcontentsline{toc}{chapter}{Results and Discussion}

In this section, we present and discuss the results of running our application. We examine the program's performance under normal conditions and its handling of various edge cases.

\subsection*{Normal Conditions}
\addcontentsline{toc}{section}{Normal Conditions}

Under normal conditions, our program performs exceptionally well. When given valid input, the program quickly loads the dataset and allows the user to perform operations on the network.

For instance, consider the following interaction:

\begin{verbatim}
	Input file path: ./data/Caltech36.txt
	[1] Get friend list
	[2] Get connection
	[3] Exit
	Enter your choice: 1
	Enter ID of Person: 10
	Person 10 has 2 friends!
	List of friends:
	341 663
\end{verbatim}

In this example, the program correctly identified the friends of person 10 from the Caltech36 dataset. This was verified by using the application \textbf{Cytoscape} and the \textbf{PathExplorer} extension.\cite{shannon2003cytoscape}

\subsection*{Edge Cases\footnote{While we're discussing edge cases here, it's worth noting that in the context of our social network graph, the edge cases aren't about the edges defining friendships. Instead, we're dealing with cases that test the boundaries of our program's behavior. So while an edge case in a social network might involve pondering whether two people who have only interacted once on a post from 5 years ago should really be considered 'friends', that's a different kind of 'edge' case entirely!}}
\addcontentsline{toc}{section}{Edge Cases}

The program also handles various edge cases gracefully. Here are some examples:

\subsubsection{Invalid File Path}

When given an invalid file path, the program displays an error message and prompts the user for a valid file path. For example:

\begin{verbatim}
Input file path: 1
Error: File not found. Please enter a valid file path.
Input file path: 
\end{verbatim}

This shows that the program is robust to errors in file input, and can recover gracefully from such situations.

\subsubsection{Invalid User ID}

When asked to list the friends of a user who does not exist in the network, the program informs the user that the ID is invalid. For example:

\begin{verbatim}
Enter your choice: 1
Enter ID of Person: 99999
Error: Person 99999 does not exist in the network.
Enter ID of Person: 
\end{verbatim}

This demonstrates the program's error handling capabilities and its ability to provide feedback to the user.

\subsubsection{Multiple Shortest Paths}

In cases where multiple shortest paths exist between two users, the program arbitrarily selects one to display. This is a result of the BFS algorithm, which does not guarantee a particular shortest path in cases where multiple paths exist.

\begin{lstlisting}[frame=lines,caption=Example of a Dataset with Multiple Paths, label=lst:MultiplePaths]
	6 6
	1 2
	2 3
	3 4
	1 5
	5 6
	6 4
\end{lstlisting}

\begin{lstlisting}[frame=lines,caption=Actual Output, label=lst:ActualOutput]
	Input file path: data/small_multiple.txt
	[1] Get friend list
	[2] Get connection
	[3] Exit
	Enter your choice: 2
	Enter ID of first person: 1
	Enter ID of second person: 4
	There is a connection from 1 to 4!
	1 is friends with 2
	2 is friends with 3
	3 is friends with 4
\end{lstlisting}
