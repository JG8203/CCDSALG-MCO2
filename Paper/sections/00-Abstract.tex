
\section*{\centering \textbf{\Large{Abstract}}}
\addcontentsline{toc}{chapter}{Abstract}
    
    \vfill
    
        This paper presents a Java-based implementation of a pseudo-social network. It utilizes a graph-based data structure and employs real-world data derived from Facebook in 2005. We utilized HashMaps and HashSets for efficient data management. This data structure represents various relationships within the social network, where each user and edge represents a user and relationship, respectively. A user's friends are stored as a HashSet, which allows for constant-time complexity during lookups and insertions, making it ideal for large datasets. The Breadth-First Search (\textbf{BFS}) algorithm is used to determine the path between two users within the network. We also provide a detailed analysis of the algorithm's implementation using its time complexity. This paper underscores the importance of choosing the appropriate data structures for efficient data representation and analysis.
    
    \vfill
    
    \textbf{Keywords:} Java, pseudo-social network, graph-based data structure, HashMaps, HashSets, real-world data, Facebook, Breadth-First Search, algorithm, time complexity, data representation, data analysis.
