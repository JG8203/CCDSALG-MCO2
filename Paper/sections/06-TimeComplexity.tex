\section*{\textbf{Time Complexity Analysis}}
\addcontentsline{toc}{chapter}{Time Complexity Analysis}

\subsection*{\textbf{Optimizing Adjacency Lists: HashMap's \textit{get()} and \textit{put()}}}
\addcontentsline{toc}{section}{Optimizing Adjacency Lists: HashMap's get() and put()}
A \textit{\textbf{Map}} is a data structure which could store key-value pairs, akin to how a physical dictionary stores a “word” and a “definition”. However unlike a physical dictionary where a single word could have multiple definitions, a \textit{\textbf{Map}} could only have a single value for each key. A \textit{\textbf{HashMap}} extends this concept by using a \textit{\textbf{Hash Function}} which “links” the key-value pair together.

Every time data (a key-value pair) is added to a \textit{\textbf{HashMap}}, the data is hashed and the key is converted into a hash-code which determines the location (bucket) in which the data is stored. We make the assumption for analysis that these keys have a uniform distribution after the hash function has placed them. This would imply each bucket has a single entry only and no hash collision would occur.

A \textit{\textbf{Hash Collision}} occurs when two or more keys are mapped to the same bucket, which could overwrite data and lead to inefficiencies within the algorithm. Given these assumptions, the \textit{get()} method with a particular key as its parameter would have a time complexity of $\mathbf{O(1)}$ due to the nature of a \textit{\textbf{HashMap}} itself. A \textit{put()} method would also have a similar time complexity because it takes a similar amount of time to “hash” any particular key-value pair, and store them into the bucket generated from the key of the data. \cite{bajracharya}

\subsection*{\textbf{Breadth-First Search (BFS) Algorithm Analysis}}
\addcontentsline{toc}{section}{Breadth-First Search (BFS) Algorithm Analysis}

\subsubsection*{Initialization}
\addcontentsline{toc}{subsection}{Initialization}
The BFS algorithm begins by initializing a parent map for backtracking, a queue, and a set for visited nodes. It adds the source node to both the queue and the visited set, and assigns its parent as null. Given that each of these steps has a constant time complexity, the initialization is $\mathbf{O(1)}$.

\subsubsection*{Queue Operations}
\addcontentsline{toc}{subsection}{Queue Operations}
Each node is added into the queue upon its first discovery. Once all its neighbors have been explored, the node is removed from the queue. This process is similar to the "coloring" of nodes in the CLRS implementation where 'White' signifies an \textbf{unvisited node}, 'Gray' signifies a node with \textbf{unvisited neighbors}, and 'Black' signifies a node with \textbf{all neighbors explored}. Since add and remove operations from a LinkedList have constant time complexity, the total time complexity of these operations is $\mathbf{O(V)}$, where $\mathbf{V}$ represents the number of nodes in the graph.\cite{cormen2001introduction}

\subsubsection*{Adjacency List Check}
\addcontentsline{toc}{subsection}{Adjacency List Check}
For each node, BFS checks its neighbors in the adjacency list to identify the unvisited nodes. These unvisited nodes are then added to the queue and the visited set, and their parents are set in the parent map. The overall time complexity for this process is equivalent to the summation of all adjacency list sizes, represented by $\mathbf{E}$, since each edge appears twice, for each endpoint.\cite{sedgewick2011algorithms}

\subsubsection*{Final Runtime for Breadth-First Search}
\addcontentsline{toc}{subsection}{Final Runtime for Breadth-First Search}
In conclusion, the sum of the time complexities of these processes results in a final time complexity of $\mathbf{O(V+E)}$, which is linear with respect to the size of the graph. This demonstrates that BFS is a notably efficient algorithm for graph traversal, as its performance scales linearly with the size of the graph.\cite{cormen2001introduction}