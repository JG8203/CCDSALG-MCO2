\section*{\textbf{Introduction}}
\addcontentsline{toc}{chapter}{Introduction}

\textbf{Graph-based data structures}, together with a wide range of traversal algorithms, serve as the foundational building blocks for numerous real-world applications. These applications span across \textbf{social networks}, route planning, as well as interactions among various biological entities. In this project, we utilized these fundamental principles to construct a program capable of representing a \textit{pseudo-social network} using \textbf{Java}.

In the digital age, \textbf{social networks} have become an integral part of our daily lives. These networks have reshaped the ways in which we communicate, share, and interact with one another. As these networks grow in size, the need for tools to analyze and understand them becomes increasingly important. This project aims to leverage the principles of graph theory to model a social network.

The objective of this project is to develop a program that could efficiently model a \textbf{social network} using graph-based data structures, as well as implement two (2) functionalities which allow users to analyze the network. To achieve these objectives, we used a real-world dataset from \textbf{Facebook} in 2005, which encapsulates the networks of numerous American universities. By analyzing this particular dataset, we aim to gain insights into the nature of \textbf{social networks}.

This paper presents an in-depth discussion of the design and implementation of our programs as well as the various operations it offers, and finally some results obtained from analyzing the dataset. We also provide the time complexity of the various algorithms within the program. The conclusions as well as the contributions of each member are also discussed within this paper.
